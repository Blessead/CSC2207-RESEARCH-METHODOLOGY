\documentclass{article}
\usepackage[margin=0.5in]{geometry}

\begin{document}

\title{ ACONCEPT PAPER OF AN  INNOVATIVE NETWORK SCANNING FRAMEWORK.}

\author{ BY: GROUP 111}
\date {13/04/2017}

\maketitle

\tableofcontents

\section{INTRODUCTION}\label{sec:into}
Competitive advantage is the ability of a network to derive abnormal protection in the competitive attackers. it is built the way the firm organizes and performs discrete activities of the value-chain. Innovation   is important in sustaining a competitive  advantage since it can represent rare, valuable and potentially inimitable sources of  strong defensive mechanisms.
Network scanning provides   defensive techniques against intrusion of attacker on a network  targeting alive hosts on a network  by finding the scanning patterns in our log (manually or automatically by IDs) will give us an indication of a probable upcoming attempts to gain unprivileged access to our syste

\section{BACKGROUND OF THE PROBLEM}\label{sec:into}
An intelligent hacker will conduct a lot of research before attempting to gain privileged access
to your systems.
If the intelligence gathered shows a poorly defended computer system, an attack will be
launched, and unauthorized access will be gained.
However, if the target is highly protected, the hacker will think twice before attempting to
break in. It will be dependent upon the tools and systems that protect the target.
Again, the key here is the amount of information he has gathered beforehand.
In the computer hacking world, intelligence gathering can be roughly divided into three major
steps:

\subsection{FOOT PRINTING}\label{sec:into}
The information collected by the hacker makes a unique footprint or a profile
of an organization security posture.
With foot printing, using rather simple tools, we gather information such as:
 
\begin{itemize}
   	  \item Administrative, technical, and billing contacts, which include employee names, email
addresses, and phone and fax numbers.
	  \item IP address range.
	\item DNS servers.
	\item Mail servers.
	\end{itemize}

\subsection{SCANNING}\label{sec:into}
The art of detecting which systems are alive and reachable via the Internet, and
what services they offer, using techniques such as ping sweeps, port scans, and operating
system identification, is called scanning.
The kind of information collected here has to do with the following:

	\begin{itemize}
   	  \item TCP/UDP services running on each system identified.
	  \item System architecture (Sparc, Alpha, x86).
	\item Specific IP addresses of systems reachable via the Internet.
	\item Operating system type.
	\end{itemize}

\section{PROBLEM STATEMENT}\label{sec:into}

Today   the number of automated scanners is constantly increasing, and as a result, more and
more attacks are successfully initiated.
In order to be better prepared, we need to fully understand the scanning tools and the
methods that these tools are using against us.


We need to identify the intruder’s behavior and understand the scanning techniques. If we
have an intrusion detection system, or planning on implementing one in the future, finding
scanning patterns in our log (manually, or automatically by the IDs) will give us an indication
of a probable upcoming attempt to gain unprivileged access to our systems.

\section{OBJECTIVES}\label{sec:into}

\subsection{Main Objective}\label{sec:into}
To   improve Integrity, reliable service delivery and confidentiality of   Information over a network to intended users.


A penetration testing over the  security  of a network by simulating an attack from malicious source.


\subsection{Other Objective}\label{sec:into}
 Enhancing   security of User and group names, System banners, Routing tables
and   SNMP  information .


\section{METHODOLOGY}\label{sec:into}

ICMP sweeps (ICMP ECHO requests)
use ICMP packets to determine whether a target IP address is alive or not, by simply
sending an ICMP ECHO request (ICMP type 8) packets to the targeted system and waiting to
see if an ICMP ECHO reply (ICMP type 0) is received. If an ICMP ECHO reply is received, it
means that the target is alive; No response means the target is down.
Blocking ICMP sweeps is rather easy, simply by not allowing ICMP ECHO requests into your
network from the void.


Broadcast ICMP
Sending ICMP ECHO request to the network or/and broadcast addresses will produce all the
information you need for mapping a targeted network in even a simpler way.
The request will be broadcast to all alive hosts on the target network, and they will send ICMP
ECHO reply to the attacker source IP after only one or two packets have been sent by him.


TCP Sweeps
The TCP connection establishment process is called “the three way handshake”, and is
combined of three segments.
When will a RESET be sent? – Whenever an arriving segment does not appear correct to the
referenced connection. Referenced connection means the connection specified by the
destination IP address and port number, and the source IP address and the port number .
Bear in mind that firewalls can spoof a RESET packet for an IP address, so TCP Sweeps may
not be reliable.


UDP Sweeps (Also known as UDP Scans)
This method relies on the ICMP PORT UNREACHABLE message, initiated by a closed UDP
port. If no ICMP PORT UNREACHABLE message is received after sending a UDP data gram
to a UDP port that we wish to examine on a targeted system, we may assume the port is
opened.

\section{OUTCOME}\label{sec:into}

A network scanning  framework  to scan for port, exploits, vulnerabilities, misconfigurations, launch DOS attack and crack passwords
A network intrusion detection/prevention system, that can detect a variety of attackers and probes, such as buffer overflows, stealth port scans, web application attacks, SMB probes, and OS fingerprinting attempts. 
To have a framework for developing and executing  exploit  code against  a remote target machine  that to say able to check whether the intended target system is susceptible to the chosen exploit ,choosing and configuring a payload(code that will be Executed on the target system upon successful  entry, for instance a remote  shell or VNC server ) and able to find the encoding technique to encode  the payload so that the intrusion prevention system will not catch the encoded payload.







\end{document}
